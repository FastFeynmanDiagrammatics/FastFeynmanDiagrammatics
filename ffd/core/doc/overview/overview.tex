\chapter{Overview of \texttt{ffd/core/}}
We briefly discuss here the general principles behind the implementation of the core parts of the library.

A \texttt{QuantumField} is a physical particle. No space position is associated to it at this stage. Products of \texttt{QuantumField} are represented by the class \texttt{QuantumFieldProduct}. At this point, we can associate to a \texttt{QuantumFieldProduct} a \texttt{Position}, and we call this object \texttt{QuantumFieldDot}. Various \texttt{QuantumFieldDot}s can be multiplied together, and we call this product a \texttt{QuantumFieldVertex}. This \texttt{QuantumFieldVertex} can represent a term in the action of the physical model or it can represent the external points of a correlation function. A sequence of \texttt{QuantumFieldVertex}es can be the basis for a \texttt{QuantumFieldGraph}, which is the set of interaction and external \texttt{QuantumFieldVertex}es.

We now need to connect the various \texttt{QuantumField}s inside a \texttt{QuantumFieldGraph}. Firstly, we bind to a \texttt{QuantumField} its graph-theoretical position by binding it in a \texttt{QuantumFieldVertexDot} object, that stores the integer positions of the \texttt{QuantumFieldDot} to which the \texttt{QuantumField} belongs. Then, we can define a graph-theoretic \texttt{FeynmanEdge} by taking a \texttt{std::pair} of \texttt{QuantumFieldVertexDot}s. We can store the numeric value of the \texttt{FeynmanEdge} in a \texttt{std::map<FeynmanEdge, Field>} that we call \texttt{FeynmanEdgeMap}. In order to save some time, it is suggested to use the symmetry properties of the expansion in order to avoid computing terms that are zero (in our language, terms belonging to different \texttt{WickFunction} blocks).

After having computed the numeric value of the allowed edges, we need to compute a correlation function by using the Wick theorem. The \texttt{QuantumField}s are inserted sequentially from the right into a \texttt{WickFunction}, that takes care of dividing the \texttt{QuantumField}s (more specifically the \texttt{QuantumFieldVertexDot}) into disconnected blocks, sorting them and computing the resulting \texttt{Sign}.

Once we have stored and sorted our \texttt{QuantumFieldVertexDot} into the \texttt{WickFunction}, we can build the \texttt{WickMatrix<Field>} by using the \texttt{FeynmanEdgeMap} to assign a numeric value to the connection inside a block of the \texttt{WickFunction}. The numeric value of each \texttt{WickMatrix<Field>} can then be easily extracted by computing a \texttt{Permutant} (a \texttt{Determinant} for fermionic particles), whose implementation is however outside the scope of \texttt{ffd/core/}.

In order to easily compute the connected part of correlation functions, we provide a complete \texttt{NilpotentPolynomial} implementation. The connected part is obtained as the ratio of two \texttt{NilpotentPolynomial}s.

