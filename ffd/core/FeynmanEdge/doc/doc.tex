\chapter{\texttt{FeynmanEdge/}}
A \texttt{QuantumFieldVertexDot} is a \texttt{std::pair<QuantumField, std::pair<int, int>>} representing a \texttt{QuantumField} and its \texttt{{Vertex, Dot}} integer positions. A \texttt{QuantumFieldPosition} is a \texttt{std::pair<QuantumField, std::any>} representing a \texttt{QuantumField} and its \texttt{Position}.
\texttt{FeynmanEdge} is a \texttt{std::pair<QuantumFieldVertexDot, QuantumFieldVertexDot>}. A \texttt{PairQuantumFieldPosition} is a \texttt{std::pair<QuantumFieldPosition, QuantumFieldPosition>} which is used by the propagators. A \texttt{FeynmanEdgeMap<Field>} is a \texttt{std::map<FeynmanEdge, Field>} that associates a numeric value to each \texttt{FeynmanEdge}.
\section{\texttt{FeynmanEdge.hpp [ffd::feynman\_edge]}}
\subsection{\texttt{QuantumFieldVertexDot}}
A \texttt{QuantumFieldVertexDot} is a \texttt{std::pair<QuantumField, std::pair<int, int>>}. It represents a \texttt{QuantumField} and its position (stored as a \texttt{std::pair} of integers) inside the \texttt{QuantumFieldGraph}. Let \texttt{QuantumFieldGraph G} be the graph we are considering, and let \texttt{G[v][d]} be one of the \texttt{QuantumFieldDot} of the graph. Then, for each \texttt{QuantumField Q} belonging to this \texttt{QuantumFieldDot}, we can associate the \texttt{std::pair<int, int> VertexDot\{v,d\}}, and we can build a \texttt{QuantumFieldVertexDot} object as \texttt{\{Q, VertexDot\}}.
\subsection{\texttt{QuantumFieldPosition}}
A \texttt{QuantumFieldPosition} is a \texttt{std::pair<QuantumField, std::any>>}. Let \texttt{QuantumFieldGraph G} be the graph we are considering, and let \texttt{G[v][d]} be one of the \texttt{QuantumFieldDot} of the graph. Then, for each \texttt{QuantumField Q} belonging to this \texttt{QuantumFieldDot}, we can retrieve its position as \texttt{std::any Pos = G[v][d].Position}, and we can build a \texttt{QuantumFieldPosition} object as \texttt{\{Q, Pos\}}.
\subsection{\texttt{FeynmanEdge}}
A \texttt{FeynmanEdge} is a \texttt{std::pair<QuantumFieldVertexDot, QuantumFieldVertexDot>}. It represent an allowed connection inside the \texttt{QuantumFieldGraph}, and the information it contains is only at the graph-theoretic level (only the integer positions of the connected \texttt{QuantumField} inside the \texttt{QuantumFieldGraph} is known).
\subsection{\texttt{PairQuantumFieldPosition}}
A \texttt{PairQuantumFieldPosition} is a \texttt{std::pair<QuantumFieldPosition, QuantumFieldPosition>}. It does not store any graph-theoretic information, it just store the \texttt{Position} of two \texttt{QuantumField} (and the \texttt{QuantumField}s themselves). It is the kind of object that can be given as argument to the functions computing propagators.
\subsection{\texttt{FeynmanEdgeMap<Field>}}
A \texttt{FeynmanEdgeMap<Field>} is a \texttt{std::map<FeynmanEdge, Field>} storing the numeric value of a \texttt{FeynmanEdge}. We use the following convention for the keys of the map: let \texttt{\{x, y\}} be a key of the \texttt{FeynmanEdgeMap}; then, \texttt{x <= y}. In this way, only one pair of connections are computed, but one should be sure to reorder the \texttt{FeynmanEdge} before any lookup of the map.
\section{\texttt{CreateFeynmanEdgeMap.hpp [ffd::user\_space]}}
\subsection{\texttt{CreateFeynmanEdgeMap(QuantumFieldGraph, drawer\_, WhichBlock\_)}}
\texttt{CreateFeynmanEdgeMap<Field>} takes as first argument a \texttt{FeynmanGraph}, as second argument a \texttt{std::function<Field(PairQuantumFieldPosition)>} (the edge drawer), and a \texttt{std::function<int(QuantumField)>} (the function that associates to each \texttt{QuantumField} a ``block'', the default is that fields of different components and statistics belong to different blocks). It returns a \texttt{FeynmanEdgeMap} that encodes the numerical \texttt{Field} value of all possible \texttt{FeynmanEdge} connections inside the \texttt{FeynmanGraph}. The possible connections are determined by the third argument (no connection between \texttt{QuantumField}s belonging to different ``blocks''). The numeric value is computed by using the second argument (the propagator, or the ``edge drawer''). It is possible to choose whether to allow self-connections inside the same vertex with the fourth argument of the function (\texttt{bool NoSelfInteraction}, the default value is \texttt{true}).

