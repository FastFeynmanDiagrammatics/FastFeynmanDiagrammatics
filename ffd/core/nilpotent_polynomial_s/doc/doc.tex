\chapter{\texttt{NilpotentPolynomial/ [ffd::nilpotent\_polynomial]}}
The \texttt{NilpotentPolynomial} class implements the arithmetic operations between nilpotent polynomials. Polynomials can be dynamically defined, summed, multiplied, etc. All the files in this folder uses the \texttt{ffd::nilpotent\_polynomial} namespace.
\section{\texttt{NilpotentPolynomial.hpp}}
\subsection{\texttt{class NilpotentPolynomial<Field = Real>}}
We consider a \texttt{NilpotentPolynomial<Field> P}. The coefficient of the polynomial are stored in a \texttt{std::vector<Field>}, and its size can be accessed with \texttt{P.size()}. The coefficients can be accessed by \texttt{P[j]}. \texttt{NotZero} is the predicate {\it ``This is not the zero polynomial''}. \texttt{NotZero}is automatically set to \texttt{true} whenever we call the non-const version of \texttt{P[j]}. The default constructor creates a zero polynomial.
\section{\texttt{operator\_multiply.hpp}}
\subsection{\texttt{NilpotentPolynomial::operator*(const NilpotentPolynomial\&)}}
This member function returns the \texttt{NilpotentPolynomial} obtained by multiplying \texttt{*this} by \texttt{fac\_}. We remind that nilpotentpolynomial multiplication is equivalent to subset convolution:
$$
p[V] = \sum_{S\subseteq V} f_1[S]\,f_2[V\setminus S]
$$
In order to loop over subsets of a given set \texttt{V} we use the following trick\footnote{Due to F. \v{S}imkovic.}. We start with a binary number \texttt{S} equal to \texttt{V}. At each following step, we update the binary number \texttt{S} with \texttt{(S-1)\&V}, where \texttt{\&} is the AND logical multibit operator. In this way, we generate all the subsets of \texttt{V}, and we can stop when we reach \texttt{S==0}. An identical strategy is used for \texttt{operator/}.
\section{\texttt{operator\_parentheses.hpp}}
\subsection{\texttt{NilpotentPolynomial::operator()(BinaryInt)}}
\texttt{NilpotentPolynomial<Field>::operator()(BinaryInt S\_)} returns a \texttt{NilpotentPolynomial<Field>} restricted to the variables represented in the binary number \texttt{S\_}.

{\it Example:}
Let $P$ be the NilpotentPolynomial corresponding to
$$
P = a+b x_1 +c x_2 +d x_1 x_2
$$
Let \texttt{S=0b10} in the binary representation (it is equal to \texttt{S=2} in base ten). Then
$$
P(S) = a + c x_1
$$
