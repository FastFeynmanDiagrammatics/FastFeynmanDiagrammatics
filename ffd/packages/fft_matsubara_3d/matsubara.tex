% Created 2OA021-01-26 Tue 22:33
% Intended LaTeX compiler: pdflatex
\documentclass[11pt]{article}
\usepackage[utf8]{inputenc}
\usepackage[T1]{fontenc}
\usepackage{graphicx}
\usepackage{grffile}
\usepackage{longtable}
\usepackage{wrapfig}
\usepackage{rotating}
\usepackage[normalem]{ulem}
\usepackage{amsmath}
\usepackage{textcomp}
\usepackage{amssymb}
\usepackage{capt-of}
\usepackage{hyperref}
\usepackage{fullpage}
\author{Riccardo Rossi}
\date{\today}
\title{}
\hypersetup{
 pdfauthor={Riccardo Rossi},
 pdftitle={},
 pdfkeywords={},
 pdfsubject={},
 pdfcreator={Emacs 27.1 (Org mode 9.3)}, 
 pdflang={English}}

\begin{document}

\tableofcontents

Suppose we have
\begin{equation}\label{}
g_{\omega_F, \omega_F'; \omega_B}
\end{equation}
where
\begin{equation}\label{}
  \omega_F := \frac{\pi(2n_F+1)}{\beta}
  \end{equation}
\begin{equation}\label{}
  \omega_B := \frac{\pi(2n_B)}{\beta}
  \end{equation}
The fermionic frequencies indexes go from $(-N_F, \dots, N_F)$, and
the bosonic frequency $(-N_B, \dots, N_B+1)$ (you need to add zero to
the $N_B+1$ coefficient if you want to use FFT).
\begin{equation}
  f(\tau_F,\tau_F',\tau_B) = \frac{1} { \beta ^ 3 } \sum_{n_F, n_F', n_B} g_
  { \omega_F, \omega_F', \omega_B } \;
  \exp\left(-i(\omega_F\tau_F+\omega_F'\tau_F'+\omega_B\tau_B)\right)
\end{equation}
\begin{equation}\label{}
\text{FFT}((a_0, a_1, \dots, a_{N-1}))= (\sum_{n} a_n, \sum_{n}
e^{2\pi i n/N} a_n,\dots, \sum_{n} e^{2\pi i n(N-1)/N} a_n)=
(\sum_{n=0}^{N-1} e^{2\pi i kn/N} a_n)_k
\end{equation}
 (we change sign of $(n, n', l)$)
\begin{equation}\label{}
  f(\tau_F, \tau'_F, \tau_B)=
  \frac{1}{ \beta ^ 3}\sum_{n_F,n_F',n_B}g_{ -\omega_F,
  -\omega_F', -\omega_B}\; e^{2\pi i
  ((n_F-1/2)\tau_F+(n_F'-1/2)\tau_F'+n_B\tau_B)/\beta}
\end{equation}
(we shift $(n_F, n_F', n_B)$)
\begin{equation}\label{}
  f(\tau_F, \tau'_F, \tau_B)
  = \frac{1}{ \beta ^ 3}\sum_{n_F=0}^{
  2N_F} \sum_{n_F'=0}^{ 2N_F}\sum_{n_B=0}^{2N_B+2} g_{ \omega_{F;-n_F + N_F},
  \omega_{F;-n_F' + N_F}, \omega_{B; -n_B + N_B }}\; e^{2\pi i
  ((n_F-N_F-1/2)\tau_F+(n_F'-N_F-1/2)\tau_F'+(n_B-N_B)\tau_B)/\beta}
\end{equation}
(we take out the constant exponential factors)
\begin{equation}\label{}
  \begin{split}
  &f(\tau_F, \tau'_F, \tau_B)=\\
    &=\exp(-2i\pi(N_F+1/2)(\tau_F+\tau_F')/\beta
    -2i\pi(N_B)\tau_B/\beta)
    )\times \\
&\times\frac{1}{ \beta ^ 3}\sum_{n_F=0}^{
  2N_F} \sum_{n_F'=0}^{ 2N_F}\sum_{n_B=0}^{2N_B+2} g_{ \omega_{F;-n_F + N_F},
  \omega_{F;-n_F' + N_F}, \omega_{B; -n_B + N_B }}\; e^{2\pi i
      (n_F\tau_F+n_F'\tau_F'+n_B\tau_B)/\beta}
    \end{split}
\end{equation}
If we choose
\begin{equation}\label{}
  \tau_F^{(k)} := \frac{\beta\,k}{2N_F}
\end{equation}
\begin{equation}\label{}
  \tau_B^{(k)} := \frac{\beta\,k}{2N_B}
\end{equation}

\begin{equation}\label{}
  \begin{split}
  &f(\tau_F^{(k_F)}, \tau_F^{(k_F')}, \tau_B^{(k_B)})=\\
      &=\exp(-2i\pi(N_F+1/2)(\tau_F+\tau_F')/\beta
    -2i\pi(N_B)\tau_B/\beta)
    )\times \\
    &\times\frac{1}{ \beta ^ 3}\sum_{n_F=0}^{
  2N_F} \sum_{n_F'=0}^{ 2N_F}\sum_{n_B=0}^{2N_B+2} g_{ \omega_{F;-n_F + N_F},
      \omega_{F;-n_F' + N_F}, \omega_{B; -n_B + N_B }}\;
    e^{2\pi i
      ((n_F k_F+n_F' k_F')/N_F+n_B k_B/N_B)}
    \end{split}
\end{equation}
\begin{equation}\label{}
  \begin{split}
  &f(\tau_F^{(k_F)}, \tau_F^{(k_F')}, \tau_B^{(k_B)})=\\
      &=\exp(-2i\pi(N_F+1/2)(\tau_F+\tau_F')/\beta
    -2i\pi(N_B)\tau_B/\beta)
    )/\beta^3\times \\
&\text{FFT}(g, (2N_F, 2N_F, 2N_B+2))
    \end{split}
\end{equation}

\section{Dual boson vertex}
\begin{equation}\label{}
\Gamma_{\omega_F,\omega_F',\omega_B} :=\langle c_{\omega_F} c^\dagger_{\omega_F+\omega_B}
c^\dagger_{\omega_F'}c_{\omega_F'+\omega_B}\rangle
\end{equation}

\end{document}
