\documentclass{article}
\usepackage{amsmath, amsfonts, fullpage}
\begin{document}
\section{Complex FFT}
We consider the discrete Fourier transform (DFT) for $k\in\{0,\dots,2^n-1\}$
$$
\tilde{f}_{k} := \sum_{j=0}^{2^n-1} e^{-2\pi ijk/2^n} \;f_j
$$
We introduce
$$
F_n^k = e^{-2\pi i k/2^n}
$$
We write
$$
\tilde{f}_k^{(n)}= \sum_{j=0}^{2^{n-1}-1} e^{-2\pi ijk/2^{n-1}} \;f_{2j}+e^{-2\pi ik/2^n}\sum_{j=0}^{2^{n-1}-1} e^{-2\pi ijk/2^{n-1}} \;f_{2j+1}
$$
We introduce the set of subsets of $\{0,\dots, 2^{n}-1\}$, $U:=2^{\{0,1, \dots, 2^{n}-1\}}$. Let
$$
\tilde{f}_k(S)=\sum_{j\in S} e^{-2\pi i jk/|S|}\;f_j
$$
One has $\tilde{f}_{k+|S|}(S)= \tilde{f}_{k}(S)$, therefore we only need to compute $k\in \{0,1,\dots, |S|-1\}$. One can write
$$
\tilde{f}_k(S)= \tilde{f}_k(E(S)) + F_n^{k\, 2^{n-|S|}}\;\tilde{f}_k(O(S))
$$
where $E(S)$ and $O(S)$ are functions that return the even and the odd part of a set $S$.
One has
$$
\tilde{f}_k(\{j\}) = f_j
$$
\section{The algorithm}
Let $n \ge 0$. At step $0$, we fill a $2^n$ vector of complex coordinates with the values of $f$:
$$
v_j^{(0)} = f_{\mathcal{R}_n\,j}
$$
where $\mathcal{R}_nj$ is the binary representation of $j$ in reverse order (e.g. $\mathcal{R}_4(0010)= 0100$).
One has
$$
v_j^{(0)} = f_0(\{\mathcal{R}_n\,j\})
$$
At step 1, we build for $j\in \{0,\dots, 2^{n-1}-1\}$:
$$
v_j^{(1)}= v_{2j}^{(0)} + v_{2j+1}^{(0)}
$$
and
$$
v_{j+2^{n-1}}^{(1)} = v_{2j}^{(0)} + F_n^{2^{n-1}}\; v_{2j+1}^{(0)}
$$
which can be written as
$$
v_{j+b 2^{n-1}}^{(1)} = v_{2j}^{(0)} + F_n^{b\,2^{n-1}}\; v_{2j+1}^{(0)}
$$
for $b\in\{0, 1\}$.

One has for $j\in\{0,\dots, 2^{n-1}-1\}$
$$
v_{j}^{(1)}=f_0(\{\mathcal{R}_nj0,\mathcal{R}_nj1 \})
$$
$$
v_{j+2^{n-1}}^{(1)}=f_{1}(\{\mathcal{R}_nj0,\mathcal{R}_nj1 \})
$$

At step $2$, we write $l\in\{0, 2^1-1\}$, for $j\in \{0,\dots, 2^{n-2}-1\}$
$$
v_{j+2^{n-2}l}^{(2)}=v_{2j+2^{n-1}l}^{(1)}+F_n^{l\,2^{n-2}}\; v_{2j+1+2^{n-1}l}^{(1)}
$$
$$
v_{j+2^{n-2}l+2^{n-1}}^{(2)}=v_{2j+2^{n-1}l}^{(1)}+F_n^{(l+2^1)\,2^{n-2}}\; v_{2j+1+2^{n-1}l}^{(1)}
$$
We can rewrite this as
$$
v_{j+2^{n-2}l+b\,2^{n-1}}^{(2)}=v_{2j+2^{n-1}l}^{(1)}+F_n^{(l+b\,2^{1})2^{n-2}}\; v_{2j+1+2^{n-1}l}^{(1)}
$$
%% One has for $j\in\{0, \dots, 2^{n-2}-1\}$
%% $$
%% v_j^{(2)}= f_0(\{\mathcal{R}j00, \mathcal{R}j01,\mathcal{R}j10,\mathcal{R}j11\})
%% $$
%% $$
%% v_{j+2^{n-2}l}^{(2)}= f_1(\{\mathcal{R}j00, \mathcal{R}j01,\mathcal{R}j10,\mathcal{R}j11\})
%% $$
%% $$
%% v_{j+2^{n-1}}^{(2)}= f_2(\{\mathcal{R}j00, \mathcal{R}j01,\mathcal{R}j10,\mathcal{R}j11\})
%% $$
%% $$
%% v_{j+2^{n-2}+2^{n-1}}^{(2)}= f_3(\{\mathcal{R}j00, \mathcal{R}j01,\mathcal{R}j10,\mathcal{R}j11\})
%% $$


In general one has for $u\in\{0,\dots, n-1\}$, $j\in\{0, \dots, 2^{n-u-1}-1\}$, $l\in\{0,\dots, 2^{u}-1\}$, $b\in\{0, 1\}$:
$$
v_{j+2^{n-u-1}l+b\,2^{n-1}}^{(u+1)}=v_{2j+2^{n-u}l}^{(u)}+F_n^{(l+b\,2^{u})2^{n-u-1}}\; v_{2j+1+2^{n-u}l}^{(u)}
$$
One has
$$
v_{j}^{(n)} = \tilde{f}_{j}
$$


\end{document}
